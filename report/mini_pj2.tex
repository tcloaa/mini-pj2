\documentclass{article}

% if you need to pass options to natbib, use, e.g.:
% \PassOptionsToPackage{numbers, compress}{natbib}
% before loading nips_2016
%
% to avoid loading the natbib package, add option nonatbib:
% \usepackage[nonatbib]{nips_2016}

% \usepackage[final]{nips_2016}

% to compile a camera-ready version, add the [final] option, e.g.:
\usepackage[final]{nips_2016}

\usepackage[utf8]{inputenc} % allow utf-8 input
\usepackage[T1]{fontenc}    % use 8-bit T1 fonts
\usepackage{hyperref}       % hyperlinks
\usepackage{url}            % simple URL typesetting
\usepackage{booktabs}       % professional-quality tables
\usepackage{amsfonts}       % blackboard math symbols
\usepackage{nicefrac}       % compact symbols for 1/2, etc.
\usepackage{microtype}      % microtypography

\title{On-hands Study of Tree-based Methods}

% The \author macro works with any number of authors. There are two
% commands used to separate the names and addresses of multiple
% authors: \And and \AND.
%
% Using \And between authors leaves it to LaTeX to determine where to
% break the lines. Using \AND forces a line break at that point. So,
% if LaTeX puts 3 of 4 authors names on the first line, and the last
% on the second line, try using \AND instead of \And before the third
% author name.

\author{
  Chenyang~Dong\\
  Department of Mathematics\\
  \texttt{cdongac@connect.ust.hk} \\
  \And
  Tsz Cheung~Lo\\
   Department of Computer Science\\
  \texttt{tcloaa@connect.ust.hk} \\
  \And
  Jiacheng~Xia\\
   Department of Computer Science\\
  \texttt{jxiaab@connect.ust.hk} \\
  %% examples of more authors
  %% \And
  %% Coauthor \\
  %% Affiliation \\
  %% Address \\
  %% \texttt{email} \\
  %% \AND
  %% Coauthor \\
  %% Affiliation \\
  %% Address \\
  %% \texttt{email} \\
  %% \And
  %% Coauthor \\
  %% Affiliation \\
  %% Address \\
  %% \texttt{email} \\
  %% \And
  %% Coauthor \\
  %% Affiliation \\
  %% Address \\
  %% \texttt{email} \\
}

\begin{document}
% \nipsfinalcopy is no longer used

\maketitle

\begin{abstract}
Intentionally left blank
\end{abstract}

\section{Introduction}
For the past few lectures we have gone through more methods for doing regression and classification analysis. Among them we decided to go for a study on tree-based methods for the second mini-project. We use tree-based methods both on the America Crime Dataset and the In-class Kaggle competitions, and we find that tree-based methods are straightforward to implement and can give satisfying performance. In this report we present the results for both tests.

The rest of the report is organized as following: Section \ref{crime-Lasso} describes the crime dataset and our previous results in project 1 using Lasso. Section \ref{crime-tree} and \ref{crime-cmp} describes the tree-based methods we used and their results, as well as some comparison and analysis between the two sets of methods. Section \ref{combodrug} describes how we used tree-based methods for combinatoric-valued drug data, their results and model comparisons. Section \ref{drug2} presents our results on the in-class Kaggle Competition: Binary DrugSensitivity2 Data. In the end there is a concluding part summarizing our findings and comments on tree-based methods.

\section{Crime dataset and previous results}
\label{crime-Lasso}

\section{Tree-based methods on crime dataset}
\label{crime-tree}

\section{Comparison and analysis}
\label{crime-cmp}

\section{Analysis of Combinatoric Drug20 using Tree-based methods}
\label{combodrug}

\section{Results of Binary DrugSensitivity2}
\label{drug2}

\section{Conclusion}
\label{con}

\subsubsection*{Acknowledgments}
\section*{References}
\small
[1] Alexander, J.A.\ \& Mozer, M.C.\ (1995) Template-based algorithms
for connectionist rule extraction. In G.\ Tesauro, D.S.\ Touretzky and
T.K.\ Leen (eds.), {\it Advances in Neural Information Processing
  Systems 7}, pp.\ 609--616. Cambridge, MA: MIT Press.
\end{document}
